% Preamble
\documentclass[a4paper,11pt]{article}
\usepackage{amsmath, amssymb, amsthm}
\usepackage[english, dutch]{babel}
\usepackage{graphicx}


\usepackage{indentfirst}



\setlength{\parindent}{12pt}			%inspringen eerste zin
\renewcommand{\baselinestretch}{1.2}	%interlinie




% titels van paragrafen springen in

% Body
\begin{document}
	\selectlanguage{english}	
	\begin{abstract}
		
	\end{abstract}
	\selectlanguage{dutch}
	
	
	
	\tableofcontents
	\newpage
	
	
	
	
	
	
	
	
	\section{Inleiding}
		De wereldwijde gezondheidscrisis die COVID-19 veroorzaakte in 2019-2020 heeft tal van middelen en energie gestuurd naar de gezondheidszorg. Elke inspanning om de verspreiding van het virus in te perken moet en zal geleverd worden. De meest effectieve methodes om dit te realiseren zijn handen wassen, mondmaskers dragen en zo veel mogenlijk afstand houden (BRON). Dit project zal voornamelijk een steentje bijdragen aan dat laaste. De Belgische veiligheidsraad en de WHO raden aan om in het bijzijn van andereminstens anderhalve meter afstand te houden. Dit blijkt voor veel mensen wel een moeilijke opdracht zeker in drukke ruimtes. Dit project heeft als doel om met behulp van camerabeelden en sensoren (microsoft kinect) aandacht te brengen voor dit punt.
		
	\section{Hoofdtekst}
	\subsection{Probleemstelling}
		Het oplossen van dit probleem zal zich hoofdzakelijk herleiden tot dataverwering en interpretatie van gegevens. Dit komt doordat de nodige informatie van de relatieve positie van personen, gehaald wordt uit camera beelden of de microsoft kinect. Dit zijn namelijk zeer grote datasets waar niet altijd even makkelijk de nodige gegevens zijn uit te halen.
		Met de microsoft kinect komen een heel aantal verschillende methodes en gegevens, die eens juist verwerkt in python, zich makkelijk verlenen aan het herwerken. Zo kan er relatief snel, afgezien van compatibiliteidsproblemen, een werkend prototype op de been gebracht worden. De mogenlijkheid bestaat natuurlijk ook om een deel van de gegevensverwerking (ZELF) te maken, vertekkend van camera beelden. Deze camerabeelden zijn voorgesteld als 3D-matrixen met voor elke pixel RGB-waardes. Met behulp van python en bijgevoegde libraries nl: opencv, numpy, (ETC.) kunnen deze ook verwerkt worden tot nuttige data.   
		  
	\subsection{Onderwerpalternatieven}
	\subsubsection{Microsoft Kinect}
	\subsubsection{Stereo imaging}
		Stereo imaging is een techniek die veel grbruikt wordt in de muziek industrie. Met stereo imaging kan een 3 dimentionele klank geproduceerd worden die zeer waarheidsgetrouw is aan de oorspronkelijke opname.(BRON) Dit wordt gerealiseerd door het gebruik van twee afzonderlijke speakers. Stereo imaging kan ook toegepast worden op videobeelden. Door het gebruik van twee parallelle camera's, op een kleine afstand van elkaar, kan op basis van het beeldverschil een diepteanalyse gemaakt worden. Deze geeft dan een schatting van de diepte relatief ten opzichte van de camera's. Tot groot voordeel van het project heeft de library opencv in python een algoritme dat deze berekening maakt en de dieptekaart geeft. Eens de dieptekaart gevonden is kan de afstand tussen twee voorwerpen berekend worden met behulp van pythagoras, $a^2+b^2=c^2 $. Hier is $a$ de afstand loodrecht aan de richting van de camera's, $b$ het verschil in dieptewaardes van de twee voorwerpen omgezet in meter, en $c$ de werkelijke afstand tussen de twee objecten.
		
	\subsubsection{Enkele Camera met AI}
		Een derde ontwerp maakte gebruik van slechts één camera. Het doel van dit ontwerp was om op een envoudige, maar accurate manier de horizontale afstand tussen twee personen te bepalen op een foto. Deze twee personen bevinden zich evenver van de camera, staan op eenzelfde lijn en kijken naar de camera (+ tekening ter illustratie invoegen). Door gebruik te maken van de python library open-cv (+referentie) kunnen we de mensen op foto herkennen, zo ook kunnen we hun ogen en gezichten herkennen. De gemiddelde afstand tussen twee mensen ogen is $6.35\text{cm}$, wanneer we dit combineren met de verhouding tussen het aantal pixels tussen de twee ogen van één van de twee personen en het aantal pixels tussen de hoofden van de twee personen, geeft dit ons een beeld van de afstand tussen de twee hoofden. Een grotere afstand tussen de twee hoofden betekent wel een grotere fout op het uiteindelijke resultaat. Dit komt onderandere doordat we aannemen dat de afstand tussen twee ogen $6.35\text{cm}$ is. Dit is een gemiddelde, geen constante.
	\subsubsection{Motivatie Keuze}
	\subsection{Methodologie}
	\subsection{Resultaten}
	\subsection{Discussie}
	\subsubsection{Voorstellen Verdere Uitwerking}
	
	
	
	\section{Besluit en Verdere Planning}
	
	
	
	\section{Vakintegratie}	% maximaal 1 pagina
	% bij welke vakken sluit dit vak aan?
	
	%- Methodiek van de informatica (B-KUL-H01B6B)
	%- Informatieoverdracht en -verwerking (B-KUL-H01D2A)
	%- Toegepaste algebra (B-KUL-H01A4B)
	%- Kansrekenen en statistiek (B-KUL-H01A6A)
	
	
	
	
	
	
	\begin{thebibliography}{10}	% APA style + hanging indentation (bv 0.5cm)!!
		
	\end{thebibliography}
	\listoffigures		% use caption
	\listoftables		% use caption
\end{document}