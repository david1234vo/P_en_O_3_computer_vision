% Preamble
\documentclass[a4paper,11pt]{article}
\usepackage{amsmath, amssymb, amsthm}
\usepackage[english, dutch]{babel}
\usepackage{graphicx}


\usepackage{indentfirst}



\setlength{\parindent}{12pt}			%inspringen eerste zin
\renewcommand{\baselinestretch}{1.2}	%interlinie




% titels van paragrafen springen in

% Body
\begin{document}
	\selectlanguage{english}	
	\begin{abstract}
		
	\end{abstract}
	\selectlanguage{dutch}
	
	
	
	\tableofcontents
	\newpage
	
	
	
	
	
	
	
	
	\section{Inleiding}
	\section{Hoofdtekst}
	\subsection{Probleemstelling}
	\subsection{Onderwerpalternatieven}
	\subsubsection{Microsoft Kinect}
	\subsubsection{Camera Kalibratie}
	\subsubsection{Enkele Camera met AI}
		Een derde ontwerp maakte gebruik van slechts één camera. Het doel van dit ontwerp was om op een envoudige, maar accurate manier de horizontale afstand tussen twee personen te bepalen op een foto. Deze twee personen bevinden zich evenver van de camera, staan op eenzelfde lijn en kijken naar de camera (+ tekening ter illustratie invoegen). Door gebruik te maken van de python library open-cv (+referentie) kunnen we de mensen op foto herkennen, zo ook kunnen we hun ogen en gezichten herkennen. De gemiddelde afstand tussen twee mensen ogen is $6.35\text{cm}$, wanneer we dit combineren met de verhouding tussen het aantal pixels tussen de twee ogen van één van de twee personen en het aantal pixels tussen de hoofden van de twee personen, geeft dit ons een beeld van de afstand tussen de twee hoofden. Een grotere afstand tussen de twee hoofden betekent wel een grotere fout op het uiteindelijke resultaat. Dit komt onderandere doordat we aannemen dat de afstand tussen twee ogen $6.35\text{cm}$ is. Dit is een gemiddelde, geen constante.
	\subsubsection{Motivatie Keuze}
	\subsection{Methodologie}
	\subsection{Resultaten}
	\subsection{Discussie}
	\subsubsection{Voorstellen Verdere Uitwerking}
	
	
	
	\section{Besluit en Verdere Planning}
	
	
	
	\section{Vakintegratie}	% maximaal 1 pagina
	% bij welke vakken sluit dit vak aan?
	
	%- Methodiek van de informatica (B-KUL-H01B6B)
	%- Informatieoverdracht en -verwerking (B-KUL-H01D2A)
	%- Toegepaste algebra (B-KUL-H01A4B)
	%- Kansrekenen en statistiek (B-KUL-H01A6A)
	
	
	
	
	
	
	\begin{thebibliography}{10}	% APA style + hanging indentation (bv 0.5cm)!!
		
	\end{thebibliography}
	\listoffigures		% use caption
	\listoftables		% use caption
\end{document}